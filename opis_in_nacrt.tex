\documentclass[a4paper,12pt]{article}
\usepackage[utf8]{inputenc}
\usepackage[T1]{fontenc}
\usepackage[slovene]{babel}
\usepackage{lmodern}  
\usepackage{amsmath,amssymb}

\newcommand{\fn}[1]{\texttt{#1}}

\title{Grafi z liho neodvisno množico velikosti 1}
\author{Mia Nardin, Tilen Žabkar}

\begin{document}
\maketitle

\section{Uvod}
Naj bo $G = (V, E)$ graf, kjer je $V$ množica vozlišč in $E$ množica povezav med njimi. Liha neodvisna množica $S \subseteq V$ je posebna vrsta neodvisne množice, kar pomeni, da nobeni dve vozlišči iz $S$ nista neposredno povezani. Poleg te lastnosti mora množica $S$ izpolnjevati še dodatni pogoj: za vsako vozlišče $v$, ki ne pripada množici $S$ (torej za vsak $v \in V \setminus S$), velja, da nima nobenega soseda v $S$ ($N(v) \cap S = \emptyset$), ali pa ima liho število sosedov v $S$ ($|N(v) \cap S| \equiv 1 \pmod{2}$). Tukaj $N(v)$ označuje odprto množico vseh sosednjih vozlišč vozlišča $v$. Največja možna moč takšne množice v grafu $G$ se imenuje liho neodvisno število grafa in jo označimo z $\alpha_{od}(G)$.\\
Barvanje grafa je močno liho, če velja, da se med vsemi sosednjimi vozlišči, vsakega vozlišča vsaka barva pojavi liho mnogokrat. Močno liho kromatično število $\chi_{so}(G)$ je najmanjše število barv, ki omogoča močno liho barvanje grafa $G$.\\
Povezava med $\alpha_{od}(G)$ in $\chi_{so}(G)$ je $$\alpha_{od}(G) \cdot \chi_{so}(G) \geqslant |G|.$$ 

\section{Opredelitev problema}
V nadaljevanju obravnavamo le grafe za katere velja $\alpha_{od}(G) = 1$. Zanimale naju bodo njihove skupne lastnosti. Primeri takih grafov so:
\begin{itemize} 
    \item $K_2$, $K_3$ in na splošno vsi $K_n$ (polni grafi), 
    \item $P_3$ (pot s tremi vozlišči), 
    \item $C_4$ (cikel s štirimi vozlišči), $C_5$ (cikel s petimi vozlišči).
\end{itemize}
Očitno je, da mora imeti vsak tak graf premer največ 2, kar pomeni, da je razdalja med katerimkoli parom vozlišč največ 2. Povezana pogoja sta še:
\begin{enumerate}
    \item Če velja $\alpha_{od}(G) = 1$, potem velja tudi $\chi_{so}(G + K_r) = 1$. \label{pogoj_1}
    \item Če je graf claw-free, potem velja, da je $\alpha_{od}(G) = 1$ natanko tedaj, ko ima graf premer največ 2. \label{pogoj_2}
\end{enumerate}

\section{Načrt dela}

Skupno delo bo potekalo prek GitHub-a in program bo napisan v Sage-u. Najprej bo potrebno definirati funkcijo \fn{alpha\_od(G)}, ki za dan graf $G$ vrne $\alpha_{od}(G)$. Funkcija bo implementirala celoštevilski linearni program iz navodil. Če se bo to izkazalo za prepočasno, bova raziskala druge načine, za hitrejše iskanje $\alpha_{od}(G)$. 

Za generiranje grafov z največ 9 vozlišči bova uporabila vgrajeno funkcijo iz Sage-a \fn{graphs(n)}. Za grafe z več kot 10 vozlišči bova uporabila vgrajeno funkcijo \fn{graphs.RandomGNP(n,\ p)} in poskusila izračunati $\alpha_{od}(G)$ le, če je \fn{G.diameter()} $\leq 2$. Smiselno bi bilo grafe sproti shranjevati in preverjati, ali je naključno generiran graf izomorfen že obstoječemu grafu z uporabo Sage metode \fn{G.is\_isomorphic(H)}.

Prav tako bo potrebno definirati funkcijo \fn{claw\_free(G)}, ki preveri ali je dan graf $G$ claw-free. Ta funkcija bo prevedla problem iskanja $\alpha_{od}(G)$ za G claw-free na preverjanje, ali je premer $G$ manjši ali enak 2.

Definirala bova tudi funkcijo \fn{chi\_{so}(G)}, ki vrne močno liho kromatično število $\chi_{so}(G)$ zaradi pogoja \ref{pogoj_1} in preverjanja lastnosti najdenih grafov.

Iz priloženih člankov v navodilih bova poiskala še več zadostnih ali potrebnih pogojev, da velja $\alpha_{od}(G) = 1$, in jih implementirala za boljše iskanje. V članku so nekateri taki grafi že našteti, npr.\ $C_3, C_4, C_5, K_p \square K_q$.

Najdene grafe $G$ bova primerjala glede na naslednje lastnosti: premer (\fn{G.diameter()}), polmer (\fn{G.radius()}), število povezav (\fn{G.size()}), število vozlišč (\fn{G.order()}), zaporedje stopenj vozlišč (\fn{G.degree\_sequence()}), ali je graf claw-free (\fn{claw\_free(G)}), $\chi_{so}(G)$ (\fn{chi\_so(G)}), ali je graf del kakšnih družin grafov (regularni grafi \fn{G.is\_regular()}, \dots), in še druge lastnosti, ki jih bova odkrila sproti.

Cilj naloge je poiskati skupne lastnosti najdenih grafov poleg že danih pogojev \ref{pogoj_1}, \ref{pogoj_2} in lastnosti, da mora biti premer manjši ali enak 2.


\end{document}