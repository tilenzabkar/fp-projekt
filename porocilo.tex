\documentclass[a4paper,12pt]{article}
\usepackage[utf8]{inputenc}
\usepackage[T1]{fontenc}
\usepackage[slovene]{babel}
\usepackage{lmodern}  
\usepackage{amsmath,amssymb}
\usepackage{booktabs}
\usepackage{graphicx}
\usepackage{float}

\newcommand{\fn}[1]{\texttt{#1}}

\begin{document}

\begin{titlepage}
    \centering

    {\Large Univerza v Ljubljani\\
    Fakulteta za matematiko in fiziko\par}
    
    \vspace{3cm}
    
    {\Huge \textbf{Grafi z liho neodvisno množico velikosti 1}\par}
    
    \vspace{3cm}
    
    {\large Avtorja:\par}
    \vspace{0.2cm}
    {\large Mia Nardin\\
    Tilen Žabkar\par}
    
    \vfill
    
    {\large Januar 2026\par}
\end{titlepage}

\tableofcontents
\newpage

\section{Uvod}
V projektu preučujemo povezane grafe, za katere velja $\alpha_{od}(G) = 1$, kjer  $\alpha_{od}(G)$ označuje velikost največje lihe neodvisne množice v grafu. Liha neodvisna množica $S$ mora izpolnjevati naslednja dva pogoja:
\begin{enumerate}
    \item $S$ je neodvisna množica (elementi znotraj $S$ niso povezani med seboj).
    \item Za vsako vozlišče $v \in V \setminus S$ velja, da je $N(v) \cap S = \emptyset$ ali $|N(v) \cap S| \equiv 1 \pmod{2}$.
\end{enumerate}
Ti grafi tvorijo zanimiv in izrazito restriktiven razred, saj lahko $S$ vsebuje natanko eno vozlišče. 

Iz literature je znano, da:
\begin{enumerate}
    \item Vsi taki grafi imajo premer največ 2.
    \item Če velja $\alpha_{od}(G) = 1$, potem velja tudi $\chi_{so}(G + K_r) = 1$.
    \item Če je graf claw-free, potem velja, da je $\alpha_{od}(G) = 1$ natanko tedaj, ko ima graf premer največ 2.
\end{enumerate}

\section{Cilj naloge}
Cilj naloge je poiskati pogoste lastnosti grafov z lastnostjo $\alpha_{od}(G) = 1$ prek potrebnih pogojev in poiskati lastnosti, ki pogosto implicirajo $\alpha_{od}(G) = 1$ prek zadostnih pogojev. To pomeni, da smo iskali pogoje, ki pomenijo majhno število protiprimerov.

\section{Generiranje grafov}

\subsection{Generiranje grafov za $|V| \leqslant 9$}
Za $n \leqslant 9$ smo generirali vse neizomorfne grafe na $n$ vozliščih in obdržali le tiste z $\alpha_{od}(G) = 1$. Opis računanja $\alpha_{od}(G)$ je v poglavju \ref{subsec:alphaod}. To smo lahko storili do $n\leqslant 9,$ ker smo predčasno izločili grafe, ki so imeli premer strogo večji od 2. Ker je to potrebna lastnost, za take grafe ni bilo potrebno računati $\alpha_{od}(G)$. Rezultati so zbrani v tabeli \ref{tab:small-graphs-alphaod1}. Slike primerov najdenih grafov z lastnostjo $\alpha_{od}(G) = 1$ so zbrane v poglavju Priloga na strani \pageref{sec:priloga}.

\begin{table}[H]
    \centering
    \begin{tabular}{cr}
        \toprule
        $n$ & Število grafov z $\alpha_{\mathrm{od}}(G) = 1$ \\
        \midrule
        1 & 1 \\
        2 & 1 \\
        3 & 2 \\
        4 & 4 \\
        5 & 11 \\
        6 & 43 \\
        7 & 266 \\
        8 & 3\,042 \\
        9 & 69\,645 \\
        \bottomrule
    \end{tabular}
    \caption{Število povezanih grafov z lastnostjo $\alpha_{\mathrm{od}}(G) = 1$ za majhne $n$.}
    \label{tab:small-graphs-alphaod1}
\end{table}

\subsection{Generiranje večjih grafov}

Za $n \geqslant 10$ je popolni pregled neizvedljiv, saj je takih grafov preveč, da bi lahko $\alpha_{od}(G)$ preverili za vsakega posebej. Odločili smo se, da bomo iskali velike grafe za $10 \leqslant n \leqslant 30.$ Iskanja velikih grafov smo se lotili na dva različna načina. 

\subsubsection{Naključni pristop}

V prvem načinu smo v vsaki iteraciji naključno generirali nov graf, kjer ima vsaka povezava na grafu verjetnost $p = 0{,}5.$ Ker v takih grafih ni bil zagotovljen pogoj $\text{diam}(G) \leqslant 2$, smo ga pred računanjem $\alpha_{od}(G)$ popravili. Popravili smo ga tako, da smo najprej poiskali dve vozlišči $v_1$ in $v_2$, ki sta oddaljeni za premer grafa. V primeru, da je teh vozlišč več, smo naključno izbrali dve. Nato smo naključno izbrali dve nesosednji vozlišči $v_3$ in $v_4$, ki se nahajata na najkrajši poti med $v_1$ in $v_2$. Te seveda vedno obstajata, saj je pogoj za ta postopek premer strogo večji od 2. Med vozliščema $v_3$ in $v_4$ smo nato dodali povezavo. Ta postopek zmanjša dolžino poti med $v_1$ in $v_2$, ki sta na razdalji $\text{diam}(G)$ in tako tudi zmanjšuje premer. To smo ponavljali, dokler ni bil premer manjši ali enak 2. Šele nato smo izračnuali $\alpha_{od}(G).$ Seveda smo pred shranjevanjem tudi preverili, če je graf izomorfen kateremu že najdenemu grafu. Za vsak $n$ od 10 do 30 vključno smo izvedli 500 takih poskusov. Pri $n = 10$ smo s takim pristopom našli 356 grafov, za $n = 30$ pa 361. Povprečno število najdenih grafov za vsak $10 \leqslant n \leqslant 30$ je nad 400.

\subsubsection{Hevristični pristop}

V drugem načinu smo samo v prvi iteraciji naključno generirali graf $G$ z verjetnostjo povezav $p = 0{,}7.$ Nato smo v vsaki iteraciji namesto nove neodvisne generacije grafa, kot smo to počeli v prejšnjem pristopu, le malce spremenili graf $G$. To smo storili tako, da smo izbrali dve naključni povezavi $e_1$ in $e_2$ v $G$ in ju zamenjali v naključno izbrani smeri. To pomeni, da smo izbrali 4 vozlišča $v_1$, $v_2$, $v_3$ in $v_4$, kjer $e_1 = \{v_1, v_2\}$ in $e_2 = \{v_3, v_4\}.$ Nato smo odstranili obstoječi povezavi $e_1$ in $e_2$ in vozlišča povezali na drugačen način. Z verjetnostjo 1/2 smo dodali povezavi $e_3 = \{v_1, v_3\}$ in $e_4 = \{v_2, v_4\}$, sicer pa smo dodali povezavi $e_3 = \{v_1, v_4\}$ in $e_4 = \{v_2, v_3\}.$ S tem postopkom smo dobili nov graf $G'$, ki pa je še vedno podoben grafu $G.$ Ponovno smo preverili in popravili premer grafa z enakim postopkom kot v naključnem pristopu, če smo z zamenjavo povezav povečali premer. Za vsak $n$ od 10 do 30 vključno, smo s tem pristopom izvedli 1000 poskusov. Za $n = 10$ smo našli 716 grafov, za $n = 30$ pa 619 grafov. Na slikah 1 in 2 lahko vidimo prvi in zadnji graf iteracije pri tem postopku pri $n = 10$ in $n = 11$.

\begin{figure}[H]
\centering
\includegraphics[width=1\textwidth]{slike/n=10.png}
\caption{Prvi in zadnji graf iteracije pri $n = 10$.}
\label{fig: n=10} 
\end{figure}

\begin{figure}[H]
\centering
\includegraphics[width=1\textwidth]{slike/n=11.png}
\caption{Prvi in zadnji graf iteracije pri $n = 11$.}
\label{fig: n=11} 
\end{figure}

Skupno smo zbrali 94456 grafov z lastnostjo $\alpha_{od}(G) = 1$.

\section{Pomožne funkcije za testiranje hipotez}

\subsection{Izračun $\alpha_{od}(G)$} \label{subsec:alphaod}
Funkcijo \fn{alpha\_{od}(G)} smo implementirali s celoštevilskim lineranim programom (CLP), ki sledi definiciji lihe neodvisne množice. Funkcija vrne velikost največje lihe neodvisne množice v grafu $G$. Testiranje na majhnih grafih je potrdilo pravilnost implementacije:
\begin{itemize}
    \item $\alpha_{od}(P_4)$ = 2,
    \item $\alpha_{od}(C_4)$ = 1,
    \item $\alpha_{od}(C_5)$ = 1.
\end{itemize}

\subsection{Izračun $\chi_{so}(G)$}
Implementirali smo še CLP za določanje najmanjšega števila barv za krepko liho barvanje grafa $G$, zaradi tesne zveze med $\alpha_{od}(G)$ in $\chi_{so}(G)$. Testiranje na majhnih grafih nam je vrnilo naslednje rezultate:
\begin{itemize}
    \item $\chi_{so}(P_4)$ = 3,
    \item $\chi_{so}(C_4)$ = 4,
    \item $\chi_{so}(C_5)$ = 5.
\end{itemize}
Opazimo, da rezultati izpolnjujejo znano neenačbo $\alpha_{od}(G) \cdot \chi_{so}(G) \geqslant |V|.$

\section{Testiranje in rezultati}

\subsection{Testiranje hipotez}
Za vsako hipotezo $H(G)$ smo testirali:
\begin{itemize}
    \item nujnost: $\alpha_{od}(G) = 1 \Rightarrow H(G)$.
    \item zadostnost: $H(G) \Rightarrow \alpha_{od}(G) = 1$.
\end{itemize}
Teste nujnosti smo izvedeli na vseh 94456 grafih, teste zadostnosti pa smo zaradi časovne zahtevnosti izvedli le na grafih do $n \leqslant 8$.

\vspace{0.5em}

\noindent Testiranje smo opravili tako, da smo definirali funkcije, ki vrnejo \verb|True| v primeru, da hipoteza drži za dan graf $G$ in \verb|False| v primeru, da hipoteza ne drži. Nato smo se za teste nujnosti sprehodili čez vse najdene grafe z $\alpha_{od}(G) = 1$ in za vsakega poračunali vrednosti funkcije za dano hipotezo. Če je funkcija vrnila \verb|False|, smo graf shranili na seznam protiprimerov. Podobno smo naredili za teste zadostnosti, le da smo se sprehodili čez vse neizomorfne grafe z $|V| \leqslant 8$ zaradi časovne zahtevnosti.

\vspace{0.5em}

\noindent Predstavili bomo le nekaj rezultatov testiranja, ki so nam bili zanimivejši. Vsi rezultati se nahajajo v datoteki \verb|text_results.txt| na repozitoriju.

\subsection{Rezultati testiranja}

\subsubsection*{H: Radij grafa je 1.}
Radij grafa 1 pomeni, da imamo eno osrednje vozlišče, ki je povezano z vsemi ostalimi. Zanimalo nas je, če imajo grafi z $\alpha_{od}(G) = 1$ pogosto osrednje vozlišče. Pri testu nujnosti smo dobili 86019 protiprimerov. To pomeni, da ima le relativno majhen delež grafov z $\alpha_{od}(G) = 1$ vozlišče, ki je neposredno povezano z vsemi ostalimi. Rezultat je smiselen, saj pogoj $\alpha_{od}(G)=1$ zagotavlja premer največ 2, ne vsiljuje pa obstoja izrazito centralnega vozlišča. Vidimo, da je radij 1 prej izjema kot tipična lasnost.

\begin{figure}[H]
\centering
\includegraphics[width=1\textwidth]{slike/radij.png}
\caption{Protiprimer hipoteze: Radij grafa je 1.}
\label{fig: radij}
\end{figure}

\subsubsection*{H: Graf je regularen.}
Graf je regularen, če imajo vsa vozlišča enako stopnjo. Pri testu nujnosti smo dobili kar 94416 protiprimerov. To kaže da pogoj $\alpha_{od}(G) = 1$ ne vsiljuje simetrije stopenj grafa. Razlog za to je v tem, da je paritetni pogoj v definiciji lihe neodvisne množice lokalen in občutljiv na strukturo sosedstev, ne pa na enakomerno porazdelitev stopenj po celotnem grafu. Regularnost torej ni tipična niti nujna lastnost grafov z $\alpha_{od}(G) = 1$. Velja pa, da večina regularnih grafov ima lastnost $\alpha_{od}(G) = 1$, namreč od 33 preverjenih grafov pri testu zadostnosti smo našli le 11 protiprimerov.

\begin{figure}[H]
\centering
\includegraphics[width=1\textwidth]{slike/regularen.png}
\caption{Protiprimer hipoteze: Graf je regularen.}
\label{fig: regularen}
\end{figure}

\subsubsection*{H: Graf nima dvojčkov.}
Graf nima dvojčkov (angl.\ is twin-free), če ne obstajata dve vozlišči z enakim sosedstvom. Pri testu nujnosti smo dobili 13406 protiprimerov. Odsotnost dvojčkov namreč pomeni, da ob povečanju lihe neodvisne množice hitro kršimo pogoj lihosti. Torej lahko trdimo, da večina grafov z $\alpha_{od}(G) = 1$ nima dvojčkov.

\begin{figure}[H]
\centering
\includegraphics[width=1\textwidth]{slike/no_twins.png}
\caption{Protiprimer hipoteze: Graf nima dvojčkov.}
\label{fig: no_twins}
\end{figure}

\subsubsection*{H: Graf je claw-free.}
Graf je claw-free, če ne vsebuje $K_{1,3}$ kot induciran podgraf. Pri testu nujnosti smo dobili 91595 protiprimerov. To pomeni, da grafi z $\alpha_{od}(G) = 1$ zelo pogosto niso claw-free. Prisotnost claw strukture pomeni, da ima neko vozlišče tri med seboj nepovezane sosede, kar lokalno omogoča večjo razpršenost sosedstva. Rezultat potrjuje, da pogoj $\alpha_{od}(G) = 1$ ne omejuje lokalnih struktur v tolikšni meri, da bi izključeval claw-free grafe, temveč dopušča precejšnjo strukturno raznolikost grafov. Pri testu zadostnosti smo iz 1145 grafov dobili 564 protiprimerov, kar kaže na to, da claw-free ni dovolj strog pogoj, da bi za take grafe veljalo $\alpha_{od}(G) = 1$.

\begin{figure}[H]
\centering
\includegraphics[width=1\textwidth]{slike/claw_free.png}
\caption{Protiprimer hipoteze: Graf je claw-free.}
\label{fig: claw_free}
\end{figure}

\subsubsection*{H: Graf je triangle-free.}
Graf $G$ je triangle-free, če ne vsebuje trikotnikov, tj. podgrafov izomorfnih $K_3$. Pri testu nujnosti smo dobili 94442 protiprimerov, kar pomeni, da odsotnost trikotnikov ni nujna lastnost tega razreda grafov. Trikotniki so lokalni kazalnik gostote in močne povezanosti med vozlišči, vendar njihova prisotnost sama po sebi še ne zagotavlja obstoj večje lihe neodvisne množice.

\begin{figure}[H]
\centering
\includegraphics[width=1\textwidth]{slike/triangle-free.png}
\caption{Protiprimer hipoteze: Graf je triangle-free.}
\label{fig: triangle-free}
\end{figure}

\subsubsection*{H: $\lambda (G) \geqslant 2$}
Povezanost grafa $G$, označena z $\lambda (G)$, je najmanjše število povezav, katerih odstranitev graf razdeli na več komponent. Če velja $\lambda (G) \geqslant 2$, potem graf nima mostov, kar pomeni, da odstranitev katerekoli povezave ne prekine povezanosti grafa. Pri testu nujnosti smo našli 313 protiprimerov. Vsi ti protiprimeri se pojavijo pri majhnih velikostih grafa, medtem ko za $n \geqslant 10$ ni bilo nobenega protiprimera z $\lambda (G) = 1$. Tukaj je pomembno omeniti, da lahko takih grafov zaradi konstrukcije nismo našli pri večjih $n$. Če ima graf most, ta povezava ločuje graf na dva dela, ki sta povezana le prek ene povezave. Takšna konstrukcija običajno omogoča konstrukcijo lihe neodvisne množice večje od 1. Zato bi tudi pričakovali, da grafi z $\alpha_{od}(G) = 1$ praviloma nimajo mostov. Vendar rezultati pokažejo, da pri manjših grafih obstajajo izjeme, kjer kljub prisotnosti mostu globalna struktura grafa še vedno prepreči obstoj večje lihe neodvisne množice.

\begin{figure}[H]
\centering
\includegraphics[width=0.85\textwidth]{slike/Edge_connectivity_at_least_2.png}
\caption{Protiprimer hipoteze: $\lambda (G) \geqslant 2$}
\label{fig: Edge connectivity at least 2}
\end{figure}

\subsubsection*{H: $\kappa (G) \geqslant 2$}
Povezanost vozlišč grafa $\kappa (G)$ je najmanjše število vozlišč, katerih odstranitev graf razdeli na več komponent. Pri testu nujnosti dobili 434 protiprimerov. Majhno število protiprimerov kaže, da je ta lastnost za grafe z $\alpha_{od}(G) = 1$ zelo tipična, čeprav ni strogo nujna. Povezanost vozlišč $\kappa (G) \geqslant 2$ pomeni, da graf nima artikulacijskih vozlišč, kar nakazuje na močno globalno povezanost. Ker imajo grafi z $\alpha_{od}(G) = 1$ majhen premer in so praviloma gosti, je pričakovano, da odstranitev enega samega vozlišča pogosto ne razbije grafa. Kljub temu pri manjših ali specifično strukturiranih grafih obstajajo izjeme, kjer lokalna struktura še vedno prepreči obstoj večje lihe neodvisne množice, čeprav graf vsebuje artikulacijsko vozlišče.

\begin{figure}[H]
\centering
\includegraphics[width=1\textwidth]{slike/Vertex_connectivity_at_least_2.png}
\caption{Protiprimer hipoteze: $\kappa (G) \geqslant 2$}
\label{fig: Vertex connectivity at least 2}
\end{figure}

\subsubsection*{H: $\alpha (G) \leqslant \frac{n}{2}$}
Z $\alpha (G)$ je označena velikost največje neodvsine množice vozlišč. Pri testu nujnosti smo dobili le 243 protiprimerov. To kaže, da je ta pogoj v veliki večini primerov izpolnjen in zato dobro opisuje tipično strukturo grafov z $\alpha_{od}(G) = 1$ čeprav ni strogo nujen. Majhno število protiprimerov je skladno z dejstvom, da so ti grafi praviloma gosti in imajo majhen premer, kar omejuje velikost neodvisnih množic. Kljub temu obstajajo izjeme, kjer globalna gostota grafa še vedno dopušča relativno veliko neodvisno množico, ki pa zaradi paritetnega pogoja ne vodi do večje lihe neodvisne množice. Rezultat tako ponovno poudarja razliko med lastnostima $\alpha (G)$ in $\alpha_{od} (G).$

\begin{figure}[H]
\centering
\includegraphics[width=1\textwidth]{slike/alpha(G)<=2.png}
\caption{Protiprimer hipoteze: $\alpha (G) \leqslant \frac{n}{2}$}
\label{fig: alpha(G) <= 2} 
\end{figure}

\subsubsection*{H: $\alpha (G) = 1$}
Pri tej hipotezi, smo pri testu nujnosti dobili 94447 protiprimerov, kar pomeni, da ta pogoj v veliki večini primerov ni izpolnjen. To jasno pokaže, da lastnost $\alpha_{od}(G) = 1$ nikakor ne implicira trivialne neodvisne strukture grafa. Čeprav gre pogosto za zelo goste grafe z majhnim premerom, ti lahko še vedno vsebujejo večje neodvisne množice, ki pa zaradi paritetnega pogoja ne ustrezajo definiciji lihe neodvisne množice.

\begin{figure}[H]
\centering
\includegraphics[width=1\textwidth]{slike/alpha(G)=1.png}
\caption{Protiprimer hipoteze: $\alpha (G) = 1$}
\label{fig: alpha(G) = 1} 
\end{figure}

\subsubsection*{H: $\chi_{so}(G) = |V|$}
$\chi_{so}(G)$ je najmanjše število barv, s katerimi lahko graf $G$ pobarvamo tako, da sosednja vozlišča nimajo iste barve in da za vsako vozlišče $v$ in vsako uporabljeno barvo velja, da se ta barva v sosedstvu vozlišča $v$ pojavi liho mnogokrat ali pa se sploh ne. Naš pogoj tako zahteva, da mora vsako vozlišče imeti svojo barvo. Zaradi časovne zahtevnosti CLP, smo testiranje za nujnost in zadostnost izvedeli le na grafih do 6 vozlišč. V obeh primerih nismo dobili nobenega protiprimera. Torej za grafe z $n \leqslant 6$ velja $\alpha_{od}(G) = 1$ natanko tedaj, ko $\chi_{so}(G) = |V|$. To je očitno, saj velja neenačba $\alpha_{od}(G) \cdot \chi_{so}(G) \geqslant |V|$, kar pa nam pove, da če je $\alpha_{od}(G) = 1$, potem nujno velja  $\chi_{so}(G) \geqslant |V|$. Ker pa je $|V|$ očitno zgornja meja za $\chi_{so}(G)$ sledi $\chi_{so}(G) = |V|$.

\subsubsection*{H: Povprečna stopnja grafa je vsaj $0{,}7 \cdot n$.}
Povprečna stopnja grafa je $\overline{d}(G) = \frac{2|E|}{|V|}$. Pri testu nujnosti smo dobili 94177 protiprimerov. Večina grafov je torej dovolj povezana, da ohranijo majhen premer in preprečijo obstoj večje lihe neodvisne množice, hkrati pa imajo nižjo povprečno stopnjo, kot bi zahtevala meja $0,7 \cdot n$.

\begin{figure}[H]
\centering
\includegraphics[width=1\textwidth]{slike/avg_degree.png}
\caption{Protiprimer hipoteze: Povprečna stopnja grafa je vsaj $0,7 \cdot n$.}
\label{fig: avg_degree}
\end{figure}

\subsubsection*{H: Gostota grafa je med 0,4 in 0,6.}
Gostota grafa je definirana kot $2 |E| \, / \, n (n-1).$ Pri hipotezi, da ima graf z lastnostjo $\alpha_{od}(G) = 1$ gostoto med 0,4 in 0,6, smo pri testu nujnosti dobili 37632 protiprimerov. To pomeni, da tak interval gostote ni nujna lastnost tega razreda grafov, je pa zelo pogost. Čeprav so grafi z $\alpha_{od}(G) = 1$ praviloma razmeroma gosti in imajo majhen premer, se njihova gostota lahko precej razlikuje in ni omejena na ozek interval. Obstajajo tako grafi, ki so bistveno redkejši, kot tudi grafi, ki so skoraj polni, pa kljub temu izpolnjujejo pogoj $\alpha_{od}(G) = 1$. Rezultat kaže, da lastnost $\alpha_{od}(G) = 1$ določa predvsem strukturne in paritetne omejitve grafa, ne pa njegove natančne gostote.

\begin{figure}[H]
\centering
\includegraphics[width=1\textwidth]{slike/density.png}
\caption{Protiprimer hipoteze: Gostota grafa je med 0,4 in 0,6.}
\label{fig: density} 
\end{figure}

\subsubsection*{H: Graf ima Hamiltonov cikel.}
Graf $G$ ima Hamiltonov cikel, če vsebuje cikel, ki obišče vsako vozlišče natanko enkrat. Pri testu nujnosti smo našli 961 protiprimerov. Odsotnost protiprimerov za večje grafe je lahko posledica konstrukcije. Hamiltonskost zahteva zelo močno globalno povezanost grafa. Ker smo že pokazali, da imajo grafi z $\alpha_{od}(G) = 1$ premer največ 2 in so praviloma zelo gosti, je pričakovano, da večina takih grafov vsebuje Hamiltonov cikel. Gostota in majhne razdalje namreč ustvarijo veliko alternativnih poti med vozlišči, kar olajša obstoj cikla, ki obišče vsa vozlišča. Kljub temu Hamiltonskost ni nujna lastnost, saj obstajajo grafi, ki so sicer dovolj gosti in imajo majhen premer, vendar zaradi lokalnih strukturnih ovir ne vsebujej Hamiltonovega cikla. Takšni primeri se pojavijo predvsem pri manjših grafih.

\begin{figure}[H]
\centering
\includegraphics[width=1\textwidth]{slike/Hamiltonian.png}
\caption{Protiprimer hipoteze: Graf ima Hamiltonov cikel.}
\label{fig: Hamiltonian} 
\end{figure}

Zanimivo je, da se lastnost, da je graf Eulerjev, pri testu nujnosti obnaša bistveno slabše, saj smo našli kar 93853 protiprimerov. Graf je Eulerjev, če vsebuje Eulerjev cikel, torej zaprt sprehod, ki vsako povezavo obišče natanko enkrat. Za povezan graf je to ekvivalentno pogoju, da imajo vsa vozlišča sodo stopnjo. Razlog za veliko večje število protiprimerov je v naravi obeh lastnosti. Hamiltonskost je globalna lastnost, ki zahteva obstoj cikla skozi vsa vozlišča, kar je povezano z lastnostmi kot je gostota grafa in majhen premer. Eulerjevost pa je povsem lokalna paritetna lastnost, saj je odvisna izključno od sode stopnje vozlišč. Graf je lahko zelo gost in močno povezan, pa kljub temu ni Eulerjev, če ima le nekaj vozlišč lihe stopnje. 

\section{Zaključek}
V nalogi smo preučili nekaj strukturnih lastnosti grafov z lastnostjo $\alpha_{od}(G) = 1$ ter testirali vrsto hipotez. Rezultati kažejo, da gre za razred grafov z zelo omejenimi globalnimi lastnostmi, kot sta majhen premer in praviloma visoka povezanost, vendar hkrati z veliko lokalno in stopenjsko raznolikostjo. Večina klasičnih lastnosti, kot so regularnost, odsotnost trikotnikov ali claws, ter natančno določena gostota, se je izkazala za nenujne. Po drugi strani majhno število protiprimerov pri hipotezah o povezanosti grafa, povezanosti vozlišč, odsotnosti dvojčkov in velikosti neodvisne množice potrjuje, da te lastnosti dobro opisujejo tipično strukturo grafov z $\alpha_{od}(G) = 1$.

\section{Priloga}
\label{sec:priloga}

\begin{figure}[H]
\centering
\includegraphics[width=0.3\textwidth]{slike/n_2.png}
\caption{Vsi grafi z $\alpha_{od}(G) = 1$ za n = 2.}
\label{fig: n_2}
\end{figure}

\begin{figure}[H]
\centering
\includegraphics[width=1\textwidth]{slike/n_3.png}
\caption{Vsi grafi z $\alpha_{od}(G) = 1$ za n = 3.}
\label{fig: n_3}
\end{figure}

\begin{figure}[H]
\centering
\includegraphics[width=1\textwidth]{slike/n_4.png}
\caption{Vsi grafi z $\alpha_{od}(G) = 1$ za n = 4.}
\label{fig: n_4}
\end{figure}

\begin{figure}[H]
\centering
\includegraphics[width=1\textwidth]{slike/n_5.png}
\caption{Vsi grafi z $\alpha_{od}(G) = 1$ za n = 5.}
\label{fig: n_5}
\end{figure}

\begin{figure}[H]
\centering
\includegraphics[width=1\textwidth]{slike/n_6.png}
\caption{Primeri grafov za n = 6.}
\label{fig: n_6}
\end{figure}

\begin{figure}[H]
\centering
\includegraphics[width=1\textwidth]{slike/n_7.png}
\caption{Primeri grafov za n = 7.}
\label{fig: n_7}
\end{figure}

\begin{figure}[H]
\centering
\includegraphics[width=1\textwidth]{slike/n_8.png}
\caption{Primeri grafov za n = 8.}
\label{fig: n_8}
\end{figure}

\end{document}