\documentclass[a4paper,12pt]{article}
\usepackage[utf8]{inputenc}
\usepackage[T1]{fontenc}
\usepackage[slovene]{babel}
\usepackage{lmodern}  
\usepackage{amsmath,amssymb}
\usepackage{booktabs}
\usepackage{graphicx}
\usepackage{float}

\newcommand{\fn}[1]{\texttt{#1}}

\begin{document}

\begin{titlepage}
    \centering

    {\Large Univerza v Ljubljani\\
    Fakulteta za matematiko in fiziko\par}
    
    \vspace{3cm}
    
    {\Huge \textbf{Grafi z liho neodvisno množico velikosti 1}\par}
    
    \vspace{3cm}
    
    {\large Avtorja:\par}
    \vspace{0.2cm}
    {\large Mia Nardin\\
    Tilen Žabkar\par}
    
    \vfill
    
    {\large December 2025\par}
\end{titlepage}

\tableofcontents
\newpage

\section{Uvod}
V projektu preučujemo povezane grafe, za katere velja $\alpha_{od}(G) = 1$, kjer  $\alpha_{od}(G)$ označuje velikost največje lihe neodvisne množice v grafu. Liha neodvisna množica $S$ mora izpolnjevati naslednja dva pogoja:
\begin{enumerate}
    \item $S$ je neodvisna množica (elementi znotraj $S$ niso povezani med seboj).
    \item Za vsako vozlišče $v \in V \setminus S$ velja, da je $N(v) \cap S = \emptyset$ ali $|N(v) \cap S| \equiv 1 \pmod{2}$.
\end{enumerate}
Ti grafi tvorijo zanimiv in izrazito restriktiven razred, saj iz definicije sledi, da lahko $S$ vsebuje največ eno vozlišče. Zanimalo nas je, kakšne strukturne značilnosti so skupne vsem grafom, za katere je $\alpha_{od}(G) = 1$ in katerim drugim lastnostim so ekvivalentni ali vsaj nujni pogoji.\\
Iz literature je znano, da:
\begin{enumerate}
    \item Vsi taki grafi imajo premer največ 2.
    \item Če velja $\alpha_{od}(G) = 1$, potem velja tudi $\chi_{so}(G + K_r) = 1$.
    \item Če je graf claw-free, potem velja, da je $\alpha_{od}(G) = 1$ natanko tedaj, ko ima graf premer največ 2.
\end{enumerate}

\section{Opis problema in cilj raziskave}
Cilj je bil identificirati nujne in zadostne pogoje za grafe z $\alpha_{od}(G) = 1$.
Izvedli smo dve vrsti preizkušanja:
\begin{enumerate}
    \item popolno generiranje vseh neizomorfnih grafov za $n \leqslant 9$,
    \item verjetnostno generiranje za $10 \leqslant n \leqslant 30$.
\end{enumerate}

\section{Metode in implementacija}

\subsection{Izračun $\alpha_{od}(G)$}
Funkcijo \fn{alpha\_{od}(G)} smo implementirali s celoštevilskim lineranim programom (CLP), ki sledi definiciji lihe neodvisne množice. Funkcija vrne velikost največje lihe neodvisne množice v grafu $G$. Testiranje na majhnih grafih je potrdilo pravilnost implementacije:
\begin{itemize}
    \item $\alpha_{od}(P_4)$ = 2,
    \item $\alpha_{od}(C_4)$ = 1,
    \item $\alpha_{od}(C_5)$ = 1.
\end{itemize}

\subsection{Izračun $\chi_{so}(G)$}
Implementirali smo še CLP za določanje najmanjšega števila barv za krepko liho barvanje grafa $G$. Testiranje na majhnih grafih nam je vrnilo naslednje rezultate:
\begin{itemize}
    \item $\chi_{so}(P_4)$ = 3,
    \item $\chi_{so}(C_4)$ = 4,
    \item $\chi_{so}(C_5)$ = 5,
    \item $\chi_{so}(K_{3,4})$ = 3.
\end{itemize}
Rezultati izpolnjujejo neenačbo $\alpha_{od}(G) \cdot \chi_{so}(G) \geqslant |V|.$

\subsection{Generiranje grafov}
Za $n \leqslant 9$ smo generirali vse neizomorfne grafe in obdržali le tiste z $\alpha_{od}(G) = 1$. Rezultati so zbrani v spodnji tabeli.

\begin{table}[ht]
    \centering
    \label{tab:small-graphs-alphaod1}
    \begin{tabular}{cr}
        \toprule
        $n$ & Število grafov z $\alpha_{\mathrm{od}}(G) = 1$ \\
        \midrule
        1 & 1 \\
        2 & 1 \\
        3 & 2 \\
        4 & 4 \\
        5 & 11 \\
        6 & 43 \\
        7 & 266 \\
        8 & 3\,042 \\
        9 & 69\,645 \\
        \bottomrule
    \end{tabular}
    \caption{Število povezanih grafov z lastnostjo $\alpha_{\mathrm{od}}(G) = 1$ za majhne $n$.}
\end{table}

\noindent
Slike primerov grafov so zbrane v prilogi.

Za $n \geqslant 10$ je popolni pregled neizvedljiv, saj je takih grafov preveč, da bi lahko $\alpha_{od}(G)$ preverili za vsakega posebej. Zato smo generirali naključen graf
in ga popravili, da ima premer $\leqslant 2$ in šele nato poračunali $\alpha_{od}(G)$. Popravil smo ga tako, da smo poiskali dve vozlišči, ki sta oddaljeni za premer grafa, in naključno izbrali dve vozlišči, ki sta na tej poti. Nato smo med tema dvema vozliščema dodali povezavo. To smo ponavljali, dokler ni bil premer manjši ali enak 2. Nato pa smo s spreminjanjem tega grafa, dobili nov graf. Za vsak $n$ smo izvedli 500 poskusov. Skupno smo zbrali 94456 grafov z lastnostjo $\alpha_{od}(G) = 1$.

\subsection{Testiranje hipotez}
Za vsako hipotezo $H(G)$ smo testirali:
\begin{itemize}
    \item nujnost: $\alpha_{od}(G) = 1 \Rightarrow H(G)$.
    \item zadostnost: $H(G) \Rightarrow \alpha_{od}(G) = 1$.
\end{itemize}
Teste nujnosti smo izvedeli na vseh 82186 grafih, teste zadostnosti pa smo zaradi časovne zahtevnosti izvedli le na grafih do $n \leqslant 8$.

\section{Rezultati}
\subsubsection*{H: Radij grafa je 1.}
Pri testu nujnosti smo dobili 86019 protiprimerov. To pomeni, da ima le relativno majhen delež teh grafov vozlišče, ki je neposredno povezano z vsemi ostalimi. Rezultat je skladen z dejstvom, da pogoj $\alpha_{od}(G)=1$ zagotavlja premer največ 2, ne vsiljuje pa obstoja izrazito centralnega vozlišča. Grafi v tem razredu so sicer globalno zelo povezani, vendar je njihova struktura pogosto bolj uravnotežena, brez enega samega dominirajočega vozlišča, zaradi česar je radij 1 prej izjema kot tipična lastnost.

\begin{figure}[H]
\centering
\includegraphics[width=1\textwidth]{slike/radij.png}
\caption{Protiprimer hipoteze: Radij grafa je 1.}
\label{fig: radij}
\end{figure}

\subsubsection*{H: Povprečna stopnja grafa je vsaj $0,7 \cdot n$.}
Povprečna stopnja grafa je $\overline{d}(G) = \frac{2|E|}{|V|}$. Čeprav imajo ti grafi premer največ 2, kar pogosto vodi do relativno goste strukture, ta pogoj še ne zahteva zelo visoke povprečne stopnje, sorazmerne z velikostjo grafa. Obstajajo namreč grafi, ki so dovolj povezani, da ohranijo majhen premer in preprečijo obstoj večje lihe neodvisne množice, vendar so hkrati bistveno redkejši, kot bi zahtevala meja $0,7 \cdot n$. To nam pokaže tudi naš rezultat, saj smo pri testu nujnosti dobili $94177$ protiprimerov.

\begin{figure}[H]
\centering
\includegraphics[width=1\textwidth]{slike/avg_degree.png}
\caption{Protiprimer hipoteze: Povprečna stopnja grafa je vsaj $0,7 \cdot n$.}
\label{fig: avg_degree}
\end{figure}

\subsubsection*{H: Graf je regularen.}
Graf je regularen, če imajo vsa vozlišča enako stopnjo. Pri testu nujnosti smo dobili 94416 protiprimerov. To kaže da pogoj $\alpha_{od}(G) = 1$ ne vsiljuje simetrije stopenj grafa. Razlog za to je v tem, da je paritetni pogoj v definiciji lihe neodvisne množice lokalen in občutljiv na strukturo sosedstev, ne pa na enakomerno porazdelitev stopenj po celotnem grafu. Graf lahko učinkovito prepreči obstoj večje lihe neodvisne množice tudi z nehomogeno porazdelitvijo stopenj, zato regularnost ni niti tipična niti nujna lastnost grafov z $\alpha_{od}(G) = 1$.

\begin{figure}[H]
\centering
\includegraphics[width=1\textwidth]{slike/regularen.png}
\caption{Protiprimer hipoteze: Graf je regularen.}
\label{fig: regularen}
\end{figure}

\subsubsection*{H: Graf nima dvojčkov.}
Graf nima dvojčkov, če ne obstajata dve različni vozlišči z enakim sosedstvom. Pri testu nujnosti smo dobili 13406 protiprimerov. Prisotnost dvojčkov kaže na lokalno simetrijo v strukturi grafa, ki pa sama po sebi še ne omogoča konstrukcije večje lihe neodvisne množice. Rezultat zato kaže, da je lastnost $\alpha_{od}(G) = 1$ kompatibilna tudi z določeno stopnjo lokalne simetrije in ne zahteva nujno stroge razlike sosedstev med vozlišči.

\begin{figure}[H]
\centering
\includegraphics[width=1\textwidth]{slike/no_twins.png}
\caption{Protiprimer hipoteze: Graf nima dvojčkov.}
\label{fig: no_twins}
\end{figure}

\subsubsection*{H: Graf je claw-free.}
Pri testu nujnosti smo dobili $91595$ protiprimerov, kar kaže, da odsotnost induciranega podgrafa $K_{1,3}$ ni nujna lastnost tega razreda grafov. Prisotnost claw strukture pomeni, da ima neko vozlišče tri med seboj nepovezane sosede, kar lokalno omogoča večjo razpršenost sosedstva. Vendar ta lokalna konfiguracija sama po sebi še ne zagotavlja obstoja večje lihe neodvisne množice, saj paritetni pogoji v definiciji $\alpha_{od}(G)$ delujejo na ravni celotnega grafa. Rezultat potrjuje, da pogoj $\alpha_{od}(G) = 1$ ne omejuje lokalnih struktur v tolikšni meri, da bi izključeval pojav claws, temveč dopušča precejšnjo strukturno raznolikost grafov.

\begin{figure}[H]
\centering
\includegraphics[width=1\textwidth]{slike/claw_free.png}
\caption{Protiprimer hipoteze: Graf je claw-free.}
\label{fig: claw_free}
\end{figure}

\subsubsection*{H: Graf je triangle-free.}
Graf $G$ je triangle-free, če ne vsebuje trikotnikov, tj. podgrafov izomorfnih $K_3$. Pri testu nujnosti smo dobili 94442 protiprimerov, kar pomeni, da odsotnost trikotnikov ni nujna lastnost tega razreda grafov. Trikotniki so lokalni kazalnik gostote in močne povezanosti med vozlišči, vendar njihova prisotnost sama po sebi še ne omogoča konstrukcije večje lihe neodvisne množice.

\begin{figure}[H]
\centering
\includegraphics[width=1\textwidth]{slike/triangle-free.png}
\caption{Protiprimer hipoteze: Graf je triangle-free.}
\label{fig: triangle-free}
\end{figure}

\subsubsection*{H: $\lambda (G) \geqslant 2$}
Povezanost grafa $G$, označena z $\lambda (G)$, je najmanjše število povezav, katerih odstranitev graf razdeli na več komponent. Če velja $\lambda (G) \geqslant 2$, potem graf nima mostov, kar pomeni, da odstranitev katerekoli povezave ne prekine povezanosti grafa. Pri testu nujnosti smo našli 313 protiprimerov. Vsi ti protiprimeri se pojavijo pri majhnih velikostih grafa, medtem ko za $n \geqslant 10$ ni bilo nobenega protiprimera z $\lambda (G) = 1$. Tukaj je znova pomembno omeniti, da lahko takih grafov zaradi konstrukcije nismo našli pri večjih $n$. Če ima graf most, ta povezava ločuje graf na dva dela, ki sta povezana le prek ene povezave. Takšna konstrukcija običajno omogoča konstrukcijo lihe neodvisne množice večje od 1. Zato bi tudi pričakovali, da grafi z $\alpha_{od}(G) = 1$ praviloma nimajo mostov. Vendar rezultati pokažejo, da pri manjših grafih obstajajo izjeme, kjer kljub prisotnosti mostu globalna struktura grafa še vedno prepreči obstoj večje lihe neodvisne množice.

\begin{figure}[H]
\centering
\includegraphics[width=0.85\textwidth]{slike/Edge_connectivity_at_least_2.png}
\caption{Protiprimer hipoteze: $\lambda (G) \geqslant 2$}
\label{fig: Edge connectivity at least 2}
\end{figure}

\subsubsection*{H: $\kappa (G) \geqslant 2$}
Povezanost vozlišč grafa $\kappa (G)$ je najmanjše število vozlišč, katerih odstranitev graf razdeli na več komponent. Pri testu nujnosti dobili 434 protiprimerov. Majhno število protiprimerov kaže, da je ta lastnost za grafe z $\alpha_{od}(G) = 1$ zelo tipična, čeprav ni strogo nujna. Vozliščna povezanost $\kappa (G) \geqslant 2$ pomeni, da graf nima artikulacijskih vozlišč, kar nakazuje na močno globalno povezanost. Ker imajo grafi z $\alpha_{od}(G) = 1$ majhen premer in so praviloma gosti, je pričakovano, da odstranitev enega samega vozlišča pogosto ne razbije grafa. Kljub temu pri manjših ali specifično strukturiranih grafih obstajajo izjeme, kjer lokalna struktura še vedno prepreči obstoj večje lihe neodvisne množice, čeprav graf vsebuje artikulacijsko vozlišče.

\begin{figure}[H]
\centering
\includegraphics[width=1\textwidth]{slike/Vertex_connectivity_at_least_2.png}
\caption{Protiprimer hipoteze: $\kappa (G) \geqslant 2$}
\label{fig: Vertex connectivity at least 2}
\end{figure}

\subsubsection*{H: $\alpha (G) \leqslant \frac{n}{2}$}
Z $\alpha (G)$ je označena velikost največje neodvsine množice vozlišč. Pri testu nujnosti smo dobili le 243 protiprimerov. To kaže, da je ta pogoj v veliki večini primerov izpolnjen in zato dobro opisuje tipično strukturo grafov z $\alpha_{od}(G) = 1$ čeprav ni strogo nujen. Majhno število protiprimerov je skladno z dejstvom, da so ti grafi praviloma gosti in imajo majhen premer, kar omejuje velikost neodvisnih množic. Kljub temu obstajajo izjeme, kjer globalna gostota grafa še vedno dopušča relativno veliko neodvisno množico, ki pa zaradi paritetnega pogoja ne vodi do večje lihe neodvisne množice. Rezultat tako ponovno poudarja razliko med lastnostma $\alpha (G)$ in $\alpha_{od} (G).$

\begin{figure}[H]
\centering
\includegraphics[width=1\textwidth]{slike/alpha(G)<=2.png}
\caption{Protiprimer hipoteze: $\alpha (G) \leqslant \frac{n}{2}$}
\label{fig: alpha(G) <= 2} 
\end{figure}

\subsubsection*{H: $\alpha (G) = 1$}
Pri tej hipotezi, smo pri testu nujnosti dobili 94447 protiprimerov, kar pomeni, da ta pogoj v veliki večini primerov ni izpolnjen. To jasno pokaže, da lastnost $\alpha_{od}(G) = 1$ nikakor ne implicira trivialne neodvisne strukture grafa. Čeprav gre pogosto za zelo goste grafe z majhnim premerom, ti lahko še vedno vsebujejo večje neodvisne množice, ki pa zaradi paritetnega pogoja ne ustrezajo definiciji lihe neodvisne množice.

\begin{figure}[H]
\centering
\includegraphics[width=1\textwidth]{slike/alpha(G)=1.png}
\caption{Protiprimer hipoteze: $\alpha (G) = 1$}
\label{fig: alpha(G) = 1} 
\end{figure}


\subsubsection*{H: $\alpha (G^2) = 1$}
Kvadrat grafa $G^2$ je graf na isti množici vozlišč kot $G$, kjer sta dve vozlišči povezani, če je njuna razdalja v $G$ največ 2. Pogoj $\alpha (G^2) = 1$ pomeni, da v $G^2$ ne obstajata dve nepovezani vozlišči, torej je $G^2$ poln graf. Ekvivalento to pomeni, da ima graf $G$ premer največ 2. Pri testiranju nujnosti smo dobili le en protiprimer in to je graf z enim samim vozliščem, pri katerem je kvadrat grafa enak samemu sebi. 

\subsubsection*{H: $\chi_{so}(G) = n$}
$\chi_{so}(G)$ je najmanjše število barv, s katerimi lahko graf $G$ pobarvamo tako, da sosednja vozlišča nimajo iste barve in da za vsako vozlišče $v$ in vsako uporabljeno barvo velja, da se ta barva v sosedstvu vozlišča $v$ pojavi liho mnogokrat ali pa se sploh ne. Naš pogoj tako zahteva, da mora vsako vozlišče imeti svojo barvo. Zaradi časovne zahtevnosti CLP, smo testiranje za nujnost in zadostnost izvedeli le na grafih do 6 vozlišč. V obeh primerih nismo dobili nobenega protiprimera. Torej za grafe z $n \leqslant 6$ velja $\alpha_{od}(G) = 1$ natanko tedaj, ko $\chi_{so}(G) = n$. Če bi v grafu poskusili uporabiti manj kot $n$ barv, bi morali isto barvo dodeliti večim vozliščem, kar je problematično, saj kot vemo so grafi $\alpha_{od}(G) = 1$ zelo gosti in imajo majhen premer, zato se sosedstva vozlišč močno prekrivajo. Prav tako vemo, da velja neenačba $\alpha_{od}(G) \cdot \chi_{so}(G) \geqslant |V|$, kar pa nam pove, da če je $\alpha_{od}(G) = 1$, potem nujno velja  $\chi_{so}(G) \geqslant |V|$. Ker pa je $|V|$ očitno zgornja meja za $\chi_{so}(G)$ sledi $\chi_{so}(G) = |V|$.

\subsubsection*{H: Gostota grafa je med 0,4 in 0,6.}
Pri hipotezi, da ima graf z lastnostjo $\alpha_{od}(G) = 1$ gostoto med 0,4 in
0,6, smo pri testu nujnosti dobili 37632 protiprimerov. To pomeni, da tak interval gostote ni nujna lastnost tega razreda grafov. Čeprav so grafi z $\alpha_{od}(G) = 1$ praviloma razmeroma gosti in imajo majhen premer, se njihova gostota lahko precej razlikuje in ni omejena na ozek interval. Obstajajo tako grafi, ki so bistveno redkejši, kot tudi grafi, ki so skoraj polni, pa kljub temu izpolnjujejo pogoj $\alpha_{od}(G) = 1$. Rezultat kaže, da lastnost $\alpha_{od}(G) = 1$ določa predvsem strukturne in paritetne omejitve grafa, ne pa njegove natančne gostote.

\begin{figure}[H]
\centering
\includegraphics[width=1\textwidth]{slike/density.png}
\caption{Protiprimer hipoteze: Gostota grafa je med 0,4 in 0,6.}
\label{fig: density} 
\end{figure}

\subsubsection*{H: Graf ima Hamiltonov cikel.}
Graf $G$ ima Hamiltonov cikel, če vsebuje cikel, ki obišče vsako vozlišče natanko enkrat. Pri testu nujnosti smo našli 960 protiprimerov. V relativnem smislu je to majhen delež, še posebej za večje grafe, saj pri $n \geqslant 12$ imajo skoraj vsi grafi z $\alpha_{od}(G) = 1$ Hamiltonov cikel, pri večjih velikostih pa protiprimerov ni več. To je zopet lahko posledica konstrukcije. Hamiltonskost zahteva zelo močno globalno povezanost grafa. Ker smo že pokazali, da imajo grafi z $\alpha_{od}(G) = 1$ premer največ 2 in so praviloma zelo gosti, je pričakovano, da večina takih grafov vsebuje Hamiltonov cikel. Gostota in majhne razdalje namreč ustvarijo veliko alternativnih poti med vozlišči, kar olajša obstoj cikla, ki obišče vsa vozlišča. Kljub temu Hamiltonskost ni nujna lastnost, saj obstajajo grafi, ki so sicer dovolj gosti in imajo majhen premer, vendar zaradi lokalnih strukturnih ovir ne vsebujej Hamiltonovega cikla. Takšni primeri se pojavijo predvsem pri manjših grafih.

\begin{figure}[H]
\centering
\includegraphics[width=1\textwidth]{slike/Hamiltonian.png}
\caption{Protiprimer hipoteze: Graf ima Hamiltonov cikel.}
\label{fig: Hamiltonian} 
\end{figure}

Zanimivo je, da se lastnost, da je graf Eulerjev, pri testu nujnosti obnaša bistveno slabše, saj smo našli kar 81587 protiprimerov. Graf je Eulerjev, če vsebuje Eulerjev cikel, torej zaprt sprehod, ki vsako povezavo obišče natanko enkrat. Za povezan graf je to ekvivalentno pogoju, da imajo vsa vozlišča sodo stopnjo. Razlog za veliko večje število protiprimerov je v naravi obeh lastnosti. Hamiltonskost je globalna lastnost, ki zahteva obstoj cikla skozi vsa vozlišča, kar je povezano z lastnostmi kot je gostota grafa in majhen premer. Eulerjevost pa je povsem lokalna paritetna lastnost, saj je odvisna izključno od sode stopnje vozlišč. Graf je lahko zelo gost in močno povezan, pa kljub temu ni Eulerjev, če ima le nekaj vozlišč lihe stopnje. 

\section{Zaključek}
V nalogi smo sistematično preučili strukturo povezanih grafov z lastnostjo $\alpha_{od}(G) = 1$ ter testirali vrsto hipotez. Rezultati kažejo, da gre za razred grafov z zelo omejenimi globalnimi lastnostmi, kot sta majhen premer in praviloma visoka povezanost, vendar hkrati z veliko lokalno in stopnjovno raznolikostjo. Večina klasičnih lastnosti, kot so regularnost, odsotnost trikotnikov ali claws, ter natančno določena gostota, se je izkazala za ne nujne. Po drugi strani majhno število protiprimerov pri hipotezah o povezljivosti in velikosti neodvisne množice potrjuje, da te lastnosti dobro opisujejo tipično strukturo grafov z $\alpha_{od}(G) = 1$.

\newpage
\section{Priloga}

\begin{figure}[H]
\centering
\includegraphics[width=0.3\textwidth]{slike/n_2.png}
\caption{Primer grafa za n = 2.}
\label{fig: n_2}
\end{figure}

\begin{figure}[H]
\centering
\includegraphics[width=1\textwidth]{slike/n_3.png}
\caption{Primera grafov za n = 3.}
\label{fig: n_3}
\end{figure}

\begin{figure}[H]
\centering
\includegraphics[width=1\textwidth]{slike/n_4.png}
\caption{Primeri grafov za n = 4.}
\label{fig: n_4}
\end{figure}

\begin{figure}[H]
\centering
\includegraphics[width=1\textwidth]{slike/n_5.png}
\caption{Primeri grafov za n = 5.}
\label{fig: n_5}
\end{figure}

\begin{figure}[H]
\centering
\includegraphics[width=1\textwidth]{slike/n_6.png}
\caption{Primeri grafov za n = 6.}
\label{fig: n_6}
\end{figure}

\begin{figure}[H]
\centering
\includegraphics[width=1\textwidth]{slike/n_7.png}
\caption{Primeri grafov za n = 7.}
\label{fig: n_7}
\end{figure}

\begin{figure}[H]
\centering
\includegraphics[width=1\textwidth]{slike/n_8.png}
\caption{Primeri grafov za n = 8.}
\label{fig: n_8}
\end{figure}

\end{document}